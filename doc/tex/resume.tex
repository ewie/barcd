\chapter{Zusammenfassung und Ausblick}
Durch die Bildaufbereitung kann die Erkennungsrate der Barcodes erhöht werden.
Jedoch verursacht die Aufbereitung auch deutliche Performanceeinbußen.
Für ein HD-Video (1920$\times$1080 Pixel) können ohne Aufbereitung 5--6 Frames pro Sekunde verarbeit werden.
Mit Aufbereitung sind $1,\!3\unit{s}$ bis $1,\!8\unit{s}$ pro Bild nötig.

Die Implementierung ist derart gestaltet, dass zentrale Verfahren wie das Auffinden von Strichcodekandidaten, die Bildaufbereitung die Grauwertberechnung sowie das Lesen der Barcodes austauschbar sind.
Dadurch ist es möglich den Extrahierungsvorgang der Eingabe anzupassen, indem geeignetere Verfahren implementiert werden, um bessere Ergebnisse zu erzielen.

