\section{Klasse \code{Extractor}}
Der \code{Extractor} stellt die Implementierung des in \autoref{sec:extraction} beschriebenen Verfahrens.
Als Eingabe dient eine Instanz von \code{Job}.
Die Verarbeitung der Bilder erfolgt sequentiell durch wiederholten Aufruf von \code{processNextImage()}.
Dem, mittels \code{setExtractionHandler()} gesetzten, \code{ExtractionHandler} werden nach jedem verarbeiteten Bild die erzeugte \code{Extraction}-Instanz sowie das entsprechende Bild übergeben.
\elasticfigure{img/uml/uml.109}{Klassendiagramm von \code{Extractor}}
\noindent
Die einzelnen Verarbeitungsschritte des Extractors sind über Interfaces realisiert, wodurch es möglich ist das Verfahren durch alternative Implementierungen anzupassen.
Das Grauwertbild wird mittels eines \code{Grayscaler}~(\code{setGrayscaler()}) bestimmt.
\code{RegionExtractor}~(\code{setRegionExtractor()}) ist für das Auffinden von Regionen zuständig.
Mit \code{Filter<Region>}~(\code{setRegionFilter()}) ist es möglich Regionen nach bestimmten Kriterien (Größe, etc.) auszuwählen.
Für jede gewählte Region wird mit einem \code{ImageEnhancer}~(\code{setImageEnhancer()}) der entsprechende Bildausschnitt bildtechnisch aufbereitet.
Abschließend erfolgt unter Verwendung eines \code{BarcodeReader}~(\code{setBarcodeReader()}) das Lesen des Barcodes im n Bildausschnitt einer jeden Region sowie im gesamten Bild.


