\section{Klasse \code{Extractor}}
Der \code{Extractor} stellt die Implementierung des in \autoref{sec:extraction} beschriebenen Verfahrens.
Als Eingabe dient eine Instanz von \code{Job}.
Die Verarbeitung der Bilder erfolgt sequentiell durch wiederholten Aufruf der Methode \code{processNextImage}.
Dem, mit der Methode \code{setExtractionHandler} gesetzten, \code{ExtractionHandler} werden nach jedem verarbeiteten Bild die erzeugte \code{Extraction}-Instanz sowie das entsprechende Bild übergeben.
\elasticfigure{img/uml/uml.109}{Klassendiagramm von \code{Extractor}}
\noindent
Die einzelnen Verarbeitungsschritte des Extractors sind über Interfaces realisiert, wodurch es möglich ist das Verfahren durch alternative Implementierungen anzupassen.
Das Grauwertbild wird mittels eines \code{Grayscaler}~(Methode \code{setGrayscaler}) bestimmt.
\code{RegionExtractor}~(Methode \code{setRegionExtractor}) ist für das Auffinden von Regionen zuständig.
Mit \code{Filter<Region>}~(Methode \code{setRegionFilter}) ist es möglich Regionen nach bestimmten Kriterien~(Größe, etc.) auszuwählen.
Für jede gewählte Region wird mit einem \code{ImageEnhancer}~(Methode \code{setImageEnhancer}) der entsprechende Bildausschnitt bildtechnisch aufbereitet.
Abschließend erfolgt unter Verwendung eines \code{BarcodeReader}~(Methode \code{setBarcodeReader}) das Lesen des Barcodes im n Bildausschnitt einer jeden Region sowie im gesamten Bild.

\subsection{Interface \code{Grayscaler}}
Der \code{Grayscaler} berechnen das Grauwertbild eines Bildes.
Aus Performancegründen setzt die Implementierung \code{DefaultGrayscaler} keines der in \autoref{sec:grayscale} vorgestellten Verfahren um.
Stattdessen wird das Eingabebild in das, nur einen Farbkanal besitzenden, Ausgabebild gezeichnet\footnote{\href{http://docs.oracle.com/javase/6/docs/api/java/awt/Graphics2D.html\#drawImage(java.awt.image.BufferedImage, java.awt.image.BufferedImageOp, int, int)}{java.awt.Graphics2D\#drawImage()}}.

Eine Messung der Zeit, die zur Konvertierung von mehreren Bildern mit einer Größe von 1920$\times$1080 Pixel (Bildmaterial eines Full-HD-Videos) nötig ist, zeigte, dass eine Implementierung nach \autoref{sec:grayscaling} rund $100\unit{ms}$ pro Bild benötigt.
Wohingegen die Implementierung des \code{DefaultGrayscaler} mit einer Laufzeit von etwa $0\unit{ms}$ pro Bild nicht messbar war.
\elasticfigure{img/uml/uml.148}{Klassendiagramm von \code{DefaultGrayscaler}}


\subsection*{Interface \code{RegionExtractor}}
Implementationen von \code{RegionExtractor} dienen dem Auffinden von Regionen in einem Bild.
Mit \code{DefaultRegionExtractor} ist das in \autoref{sec:candidate-extraction} beschriebene Verfahren realisiert 
\elasticfigure{img/uml/uml.146}{Klassendiagramm von \code{DefaultRegionExtractor}}


\subsection{Interface \code{ImageEnhancer}}
Mit einem \code{ImageEnhancer} werden die Bildausschnitte aufbereitet um die Chancen zu erhöhen, dass der \code{BarcodeReader} einen Barcode erfolgreich lesen kann.
Die Standardimplementierung ist mit \code{DefaultImageEnhancer} gegeben.
\elasticfigure{img/uml/uml.149}{Klassendiagramm von \code{DefaultImageEnhancer}}


\subsection*{Interface \code{BarcodeReader}}
Die Funktion eines \code{BarcodeReader} liegt im Lesen von Barcodes in einem Bild.
Die Implementierung \code{DefaultBarcodeReader} verwendet ZXing~(siehe \autoref{par:zxing}) zum Dekodieren der Barcodes.
%\elasticfigure{img/uml/uml.147}{Klassendiagramm von \code{DefaultBarcodeReader}}


\begin{figure}[h]
  \centering
  \elasticgraphic{img/uml/uml.147}
  \caption{Klassendiagramm von \code{DefaultBarcodeReader}}
\end{figure}



