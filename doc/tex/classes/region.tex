\section{Klasse \code{Region}}
Instanzen der Klasse \code{Region} beschreiben Bildregionen.
Als Grundlagen dienen z.B. die zusammenhängenden Komponenten eines Bild.
Eine Region ist durch ein konvexes Polygon~(Methode \code{getConvexPolygon}) und einen Deckungsgrad~(Methode \code{getCoverage}) beschrieben.
Die Methode \code{getOrientedRectangle} erzeugt aus dem konvexen Polygon das umschließende Rechteck mit minimaler Fläche.
Im Gegensatz dazu erzeugt die Methode \code{getAxisAlignedRectangle} das achsenparallele Rechteck.

\elasticfigure{img/uml/uml.116}{Klassendiagramm von \code{Region}}

