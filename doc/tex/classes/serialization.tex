\section{Serialisierung}
Um Jobs und Extractions speichern und zwischen Anwendungen austauschen zu können ist eine Serialisierung, d.h. die Übersetzung von Objekten in eine maschinenlesbare Form sowie die Wiederherstellung jener Objekte, erforderlich.

XML stellt für diesen Zweck die ideale Lösung dar, da es ein weit verbreiteter Standard zur externen Repräsentierung von Daten darstellt.
Zudem wird an der Professur Medieninformatik im Bereich Informationretrieval bereits die XML-Datenbank eXist\footnote{\url{http://www.exist-db.org/}} eingesetzt wodurch eine XML-Lösung naheliegend erscheint.
Zusammen mit einem XML Schema (siehe \autoref{annex:xml-schema}) ist es möglich die Serialisierung weitesgehend zu automatisieren, da Werkzeuge existieren (z.B. JAXB für Java) die aus einem XML Schema Quellcode erzeugen, der die Serialisierung realisiert.

Die Grundlage der XML-Serialisierung bildet die abstrakte Klasse \code{XmlSerialzer}, die wiederum eine Realisierung des Interface \code{Serializer} ist.
Konkrete Serialisierer wie z.B. \code{XmlJobSerializer} für \code{Job} und \code{XmlExtractionSerializer} für \code{Extraction} müssen ein Objekt des entsprechenden Typs auf ein XML-Element abbilden (Methode \code{createRootElement}) bzw. aus einem XML-Element ein Objekt wieder herstellen (Methode \code{restoreModel}).
Neben Zeichenketten und Bytestreams als Ein- und Ausgabe bietet \code{XmlSerializer} zusätzlich DOM-Knoten als Quelle und Ausgabe.

Optional kann \code{XmlSerializer} die Eingabe bzw. Ausgabe validieren, indem mit der Methode \code{setSchema} das XML Schema gesetzt wird.
Das Schema kann auch über eine URL mittels der Methode \code{setSchemaLocation} verwendet werden.
Ist die URL des Schemas bekannt, so kann diese bei der Serialisierung mit ausgegeben\footnote{\url{http://www.w3.org/TR/xmlschema-1/\#xsi_schemaLocation}} werden.
Die Methode \code{setUrlContext} setzt den URL-Kontext um relative URLs (z.B. relative Dateipfade) auflösen zu können.

\elasticfigure{img/uml/uml.150}{Klassendiagram von \code{XmlSerialzer}}

