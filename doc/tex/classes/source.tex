\section{Klasse \code{Source}}
Das Klasse \code{Source} stellt eine Bildquelle dar, die ein Objekt erzeugen kann welches Bilder bereitstellt~(\code{createImageProvider()}).
Eine konkrete Bildquelle kapselt dazu die Informationen die zur Erzeugung einer Bildquelle nötig sind (z.B. die URL einer Videoresource).

\elasticfigure{img/uml/uml.119}{Klassendiagramm von \code{Source}}

\subsection*{Konkrete Implementierungen}
Insgesamt existieren 6 Realisierungen von \code{Source} die sich im Laufe der Entwicklung als sinnvoll erwiesen haben.

{
\setlength{\leftmargini}{1.5em}
\setlength{\labelsep}{\textwidth}
\begin{description}
  \item[\code{BufferedImageSource}]
    verwendet eine Liste von \code{BufferedImage}
  \item[\code{ImageCollectionSource}]
    verwendet eine Liste von URLs um Bilder zu laden
  \item[\code{ImageSequenceSource}]
    verwendet eine URL um durchnummerierte Bilder zu laden
  \item[\code{ImageSnapshotServiceSource}]
    verwendet eine URL für einen Webservice, der bei jedem Request ein neues Bild liefert (z.B. eine Webcam)
  \item[\code{VideoDeviceSource}]
    verwendet, wie auch OpenCV, eine Zahl beginnend bei 0, welche die zu verwendente Kamera identifiziert
  \item[\code{VideoFileSource}]
    verwendet eine URL um ein Video zu laden
\end{description}
}

