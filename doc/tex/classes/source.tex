\section{Klasse \code{Source}}
Das Klasse \code{Source} stellt eine Bildquelle dar, die ein Objekt erzeugen kann welches Bilder bereitstellt~(\code{createImageProvider()}).
Eine konkrete Bildquelle kapselt dazu die Informationen die zur Erzeugung einer Bildquelle nötig sind (z.B. die URL einer Videoresource).

\elasticfigure{img/uml/uml.119}{Klassendiagramm von \code{Source}}

\subsection*{Konkrete Implementierungen}
Insgesamt existieren 6 Realisierungen von \code{Source} die sich im Laufe der Entwicklung als sinnvoll erwiesen haben.
Die Klasse \code{SourceFactory} stellt Factory-Methoden zu Verfügung um die hier gelisteten Implementierungen zu erzeugen.

{
\setlength{\leftmargini}{1.5em}
\setlength{\labelsep}{\textwidth}
\begin{description}
  \item[\code{BufferedImageSource}]
    Verwendet eine geordnete Sammlung von \code{BufferedImage}-Instanzen.
    Diese Klasse ist sinnvoll wenn die Bilder bereits von einer Anwendung geladen sind.
  \item[\code{ImageCollectionSource}]
    Verwendet eine Liste von URLs um Bilder lokal oder aus einem Netzwerk (z.B. über HTTP) zu laden.
  \item[\code{ImageSequenceSource}]
    Verwendet eine URL um durchnummerierte Bilder (ebenfalls lokal oder aus einem Netzwerk) zu laden.
    Ein Anwendungsfall ist z.B. Bilder die zuvor bereits aus einem Video extrahiert wurden und daher durchnummeriert vorliegen.
  \item[\code{ImageSnapshotServiceSource}]
    Verwendet eine URL für einen Webservice, der bei jedem Request ein neues Bild liefert (z.B. eine Netzwerkwebcam).
  \item[\code{VideoDeviceSource}]
    Nutzt eine Videogerät, welches über eine Zahl, beginnend bei 0, identifiziert wird.
    Zum Lesen der Videodaten dient JavaCV\footnote{\url{http://code.google.com/p/javacv/}} um auf die C/C++-API von OpenCV\footnote{\url{http://opencv.org}} zugreifen zu können.
  \item[\code{VideoFileSource}]
    Verwendet eine URL zum laden eines Videos, welches lokal aber auch in einem Netzwerk (Zugriff über HTTP) vorliegen kann.
    Nutzt ebenfalls JavaCV zum Lesen der Daten.
\end{description}
}

