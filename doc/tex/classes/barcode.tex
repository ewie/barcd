\section{Klasse \code{Barcode}}
Die Klasse \code{Barcode} dient der Beschriebung erfolgreich gelesener Barcodes.
Ein Barcode besteht aus seinem Typ~(Methode \code{getType}), dem kodierten Text~Methode (\code{getText}), den Binärdaten~(Methode \code{getBytes}) sowie mehreren Punkten~(Methode \code{getAnchorPoints}), die zum Lesen des Barcodes dienten.
Die Punkte können z.B. der Start- und Endpunkt der Scanline für einen eindimensionalen Barcode sein oder die Mittelpunkte der Marker bei einem zweidimensionalen Barcode~(z.B. QR Code).

\elasticfigure{img/uml/uml.101}{Klassendiagramm von \code{Barcode}}

