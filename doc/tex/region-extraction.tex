\chapter{Auffinden und Extrahieren der Barcodekandidaten}
Um Regionen mit potentiellen Barcodes zu ermitteln machen wir uns den Umstand
zu Nutze, dass Barcodes einen relativ hohen Kontrast aufweisen und sich in
dieser Hinsicht von deren Umgebung abheben.

\begin{enumerate}[(i)]
  \item \textbf{Grauwertbild}
    Das Grauwertbild \( I_{lum} \) wird durch folgen einfache Formel berechnet.
    \[ I_{lum} = \frac{I_r + 2I_g + I_b}{4} \]

  \item \textbf{Kantenbild bestimmen}
    Das Kantenbild \( I_{edge} \) ergibt sich aus der pixelweisen Addition der
    absoluten Werte zweier Gradientenbilder (in horizontaler und vertikaler
    Richtung). Die beiden Gradientenbilder werden durch Faltung mit einem Kernel
    \( K \) berechnet.
    \[ I_{edge} = \Big| I_{lum} \circ K \Big|
                + \Big| I_{lum} \circ K^\top \Big|
    \]
    Wir verwenden als Kernel den Roberts-Operator
      \cite{DBLP:books/garland/Roberts63}.
    \[ K = K_{Roberts} = \left[ \begin{array}{cc}
         -1 & 0 \\
          0 & 1
       \end{array} \right]
    \]
    Dieser hat sich für unser Problem als ideal erwiesen, da er im Gegensatz zum
    gängigeren Sobel-Operator
    \[ K_{Sobel} = \left[ \begin{array}{ccc}
         -1 & -2 & -1 \\
          0 &  0 &  0 \\
          1 &  2 &  1
       \end{array} \right]
    \]
    robuster gegenüber kleinen Helligkeitsänderungen ist, welche mit dem
    Sobel-Operator eher als Kante erfasst würden.

  \item \textbf{Kanten verstärken}
    \[ I_{edge}^\bullet = I_{edge} \oplus D \]

  \item \textbf{Segmentierung}
    Das Kantenbild wird anschließend segmentiert um jedes Pixel entweder dem
    Vordergrund (Kanten) oder dem Hintergrund zuzuordnen. Wir nutzen dazu das
    Schwellwertverfahren.
    \[ I_{bin}(x,y) = \begin{cases}
         255 & I_{edge}^\bullet(x,y) > T \\
         0   & I_{edge}^\bullet(x,y) \leq T
       \end{cases}
    \]
    Als Schwellwert \( T \) dient die mittlere Helligkeit aller Pixel.
    \[ T = \frac{1}{NM} \sum_{x,y}{I_{edge}^\bullet(x,y)} \]
    
  \item \textbf{Komponenten extrahieren}
    Um alle zusammenhängenden Komponente aus \(I_{bin}\) zu extrahieren wenden
    wir das \textit{connected-component labeling} \cite[69--75]{compvis2001} an.
    Das Verfahren bestimmt für jedes Pixel aus \(I_{bin}\) desen Zugehörigkeit
    zu einer Komponente. Das Resultat ist eine Punktmenge für jede Komponente.
  
  \item \textbf{Regionen klassifizieren}
    Die extrahierten Komponenten beschreiben bisher nur Bildelemente mit
    erhöhtem Kontrast. Darunter fallen, neben den erwünschten Barcodes, auch
    Elemente wie z.B. Schrift und Symbole. Um alle potentiellen Barcodes
    auszuwählen müssen wir alle Komponente herausfiltern welche nicht als
    Barcode in Frage kommen. Dazu berechnen wir für jede Komponente deren
    konvexe Hülle \cite[6--7]{compgeom2008} über den Koordinaten der
    entsprechenden Pixel. Jetzt können wir für jede Region den Deckungsgrad \(
    \gamma \) definieren, d.h. welchen Anteil die Fläche einer Komponente an der
    Fläche \( A \) des Polygons \cite{braden1986} besitzt, das durch die
    konvexe Hülle beschrieben wird.
    \begin{align*}
      \gamma &= \frac{|C|}{A}
             & C~\hat=~\text{Punktmenge einer Komponente} \\
      A      &= \frac{1}{2} \sum_{i=0}^{n-1}{\hat x_i (\hat y_{i+1} - \hat
                                             y_{i-1})}
             & \hat x_i = x_{i \bmod n}
    \end{align*}

    \cite[4]{DBLP:conf/cvpr/DalalT05}
\end{enumerate}
