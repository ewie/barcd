\section{Extrahieren der Regionen und Barcodes}
\label{sec:extraction}
\writtenby{\dcauthornameewie}%
%
Die Extrahierung der Regionen und Barcodes bildet das Kernstück dieser Arbeit.
Das Verfahren besteht aus 3 Schritten.
%
\begin{enumerate}[(1)]
\item Regionen mit potentiellen Barcodes finden.
\item Jede gefundene Region bildtechnisch aufbereiten und den eventuellen Barcode lesen.
\item Alle lesbaren Barcodes im Ausgangsbild lesen.
Dies ermöglicht es, in Schritt~(1) nicht erfasste, Barcodes mit ins Ergebnis einzubeziehen.
\end{enumerate}
%
Zum besseren Verständnis sei im Folgenden der Ablauf in Form eines Pseudocodes gegeben.
%
\begin{algorithm}
\caption{Extrahieren von Regionen und Barcodes}
\begin{algorithmic}
\Require Graustufenbild $I$
\Ensure Menge $X$ mit Barcode-Region-Paaren
\State $X\gets\varnothing$
\State $R\gets FindCandidateRegions(I)$
\ForAll {$r\in R$}
  \State $I_r\gets GetRegionImage(I,r)$
  \State $I_r\gets ImproveImage(I_r)$
  \State $b\gets ReadBarcode(I_r)$
  \State $X\gets X\cup \{(b,r)\}$
\EndFor
\State $B\gets ReadMultipleBarcodes(I)$
\ForAll {$b\in B$}
  \If {$b\notin\{b'|(b',r')\in X\}$}
      \Comment {wenn $b$ noch nicht Teil der Ergebnismenge}
    \State $X\gets X\cup \{(b,\varnothing)\}$
  \EndIf
\EndFor
\State\Return $X$
\end{algorithmic}
\end{algorithm}

