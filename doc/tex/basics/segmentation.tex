\section*{Segmentierung}

Die Segmentierung findet Anwendung bei der Klassifizierung von Bildelementen.
Dabei wird jedes Pixel anhand eines Kriteriums einer Klasse zugeordnet.
Ein simples Verfahren ist dabei die Verwendung eines Schwellwertes.
  \[ C_I^{(n)}(x,y) = \begin{cases}[c@{\quad}r@{}c@{}l@{}]
       0      &              & \;I(x,y) & \;<    t_1     \\
       1      & t_1     \leq & \;I(x,y) & \;<    t_2     \\
       \vdots &              &   \vdots &                \\
       n-2    & t_{n-2} \leq & \;I(x,y) & \;<    t_{n-1} \\
       n-1    &              & \;I(x,y) & \;\geq t_{n-1}
     \end{cases} \]
Eine der häufigsten Anwendungen ist die Einteilung eines Bildes in zwei Klassen.
Für gewöhnlich ist dies eine Einteilung des Bildes in Vorder- und Hintergrund.
  \[ C_I^{(2)}(x,y) = \begin{cases}
       0 & I(x,y) <    t \\
       1 & I(x,y) \geq t
     \end{cases} \]
Die Berechnung des Schwellwerts $t$ ist dabei entscheidend über die Qualität der Segmentierung, d.h. dass Bildelemente korrekt klassifiziert werden.

\subsection*{Schwellwertbestimmung}
\todo[author=Erik,inline]{Schwellwertbestimmung}

\paragraph{Mittelwert}
Nutzt den Mittelwert der Helligkeit aller Bildpunkte als Schwellwert.
\todo[author=Erik]{Mittelwert Vor- \& Nachteile}

\paragraph{Otsu's Method}
\textsc{Otsu}'s Method ist ein histogrammbasiertes Verfahren, dass eine bimodale Helligkeitsverteilung\footnote{Eine bimodale Helligkeitsverteilung besitzt zwei Maxima die den Helligkeitswerten des Hintergrund bzw. Vordergrund entsprechen. Der optimale Schwellwert liegt im Tal zwischen den Maxima.} voraussetzt.
Der Algorithmus sucht den Schwellwert, der die Varianz innheralb der beiden Klassen minimiert.

\paragraph{Balanced Histogram Thresholding}
Das Balanced Histogram Thresholding erwartet, wie \textsc{Otsu}'s Method, eine bimodale Helligkeitsverteilung.
Das Verfahren ermittelt den Schwellwert derart, dass dieser das Histogramm in zwei gleich "`schwere"' Hälften teilt.
Als Maß für das Gewicht einer Hälfte dient die Summe der absoluten Häufigkeiten aller Klassen der betrachteten Hälfte.
Von der schwereren Häfte werden solange Klassen entfernt bis nur noch eine Klasse betrachtet wird, welche dem gesuchten Schwellwert entspricht.

\subsection*{Schwellwertanwendung}
\todo[author=Erik,inline]{Schwellwertanwendung}

\paragraph{global}
Der Schwellwert wird für das gesamte Bild einmalig berechnet und auf alle Bildpunkte angewandt.
Die Verwendung eines einzelnen Schwellwertes für das gesamte Bild funkioniert nur sofern das Bild gleichmäßtig Belichtet ist.
Ansonsten kann es vorkommen, dass Bildteile die zu gering belichtet sind unterhalb des Schwellwertes liegen und somit als Hintergrund klassifiziert werden.

\paragraph{lokal}
Unterteilt das Bild in disjunkte Regionen und ermittelt jeweils einen eigenen Schwellwert, mit dem der jeweilige Bereich segmentiert wird.
Die negativen Effekte des globalen Thresholding auf das Gesamtbild werden vermieden, können innerhalb einer Region aber dennoch auftreten.
Wodurch es an den Regionsgrenzen zu unterschiedlichen Klassifikationen kommen kann.

\paragraph{dynamisch}
Ermittelt für jeden Bildpunkt einen Schwellwert unter Betrachtung der Umgebung des Bildpunktes.
Anhand des Schwellwertes wird der Bildpunkt einer Klasse zugeordnet.
Vermeidet die Fehler durch ein ungleiche Beleuchtung, wenn eine Pixelumgebung gewählt werden kann, die einer gleichmäßigen Beleuchtung unterliegt.

