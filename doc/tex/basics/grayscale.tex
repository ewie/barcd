\section*{Graustufen}
\writtenby{\dcauthornameewie}%
In der Regel reicht es für die Bildanalyse aus Strukturen anhand ihrer Helligkeit und daraus resultierenden Kontrasten zu erkennen.
Daher ist es nicht nötig für die Varbeitung eventuell farbige Ausgangsbilder (z.B. \textit{sRGB} oder \textit{YUV}) zu betrachten.
Deshalb ist es angebracht die Helligkeit $Y$ eines Originalbildes zu extrahieren wodurch sich ein Graustufenbild ergibt, welches die Helligkeit eines jeden Bildpunktes kodiert.
Im Falle von \textit{YUV} ist diese Umwandlung trivial da die Helligkeit direkt in den Bilddaten kodiert ist.
Liegt jedoch \textit{sRGB} vor so ist eine etwas umfangreichere Umrechnung erforderlich.

CIE XYZ definiert die Umrechung wie folgt, wobei hier der Fokus darauf liegt die vom Menschen wahrgenommene Helligkeit zu extrahieren\todo[author=\dcauthornameewie]{Quelle}.
  \[ Y = 0.299 R + 0.587 G + 0.114 B \]
Soll die Konvertierung möglichst schnell erfolgen\footnote{\href{https://code.google.com/p/zxing/source/browse/trunk/core/src/com/google/zxing/RGBLuminanceSource.java?spec=svn2633&r=2394\#57}{ZXing RGBLuminanceSource.java r2394 L57 (code.google.com)}} wird meist auf eine einfachere Variante zurückgegriffen, welche die Komponente $G$, im Vergleich zu $R$ und $B$, doppelt gewichtet\footnote{diese Gewichtung wird übrigens auch von allen Digitalkameras angewandt deren CCD-Sensor als \textsc{Bayer}-Matrix aufgebaut ist}.
  \[ Y = \frac{1}{4}(R + 2 G + B) \]
Da die Faktoren allesamt Zweier-Potenzen sind kann die Formel ausschließlich durch 3 Additionen sowie 1 Rechtsshift umgesetzt werden.
  \[ Y = (R + G + G + B) >> 2 \]
