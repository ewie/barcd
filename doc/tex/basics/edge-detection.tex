\section{Kantenerkennung}
\writtenby{\dcauthornameewie}%
Kanten in einem Bild zeigen sich durch abrupte Helligkeitsänderungen zwischen benachbarten Pixeln.
Um die Helligkeitsänderung für jedes Pixel zu bestimmen muss der Gradient der Helligkeitsfunktion $\nabla I$ eines Bildes berechnet werden.
Das gängigste Verfahren zur Gradientenbestimmung ist die Faltung des Bildes $I$ mit einer Faltungsmatrix $K$ (auch genannt Kernel).
  \[ (I\circ K)(x,y) =
       \sum_{(u,v)\in m\times n}
       I\left(x+u-\frac{m}{2},y+v-\frac{n}{2}\right)K_{u,v}
       \quad K\in\mathbb{R}^{m\times n} \]
Die Berechnung eines einzelnen Gradientenbildes reicht jedoch nicht aus um alle Kanten eines Bildes zu erfassen, da die Faltungsmatrix nur Helligkeitsänderungen entlang einer Dimension erfasst.
Deshalb ist ein zweites Gradientenbild unter Verwendung einer Faltungsmatrix, die Helligkeitsänderungen entlang einer zweiten Dimension erfasst, erforderlich.
In der Regel nutzt man dafür die Transponierte der ersteren Faltungsmatrix.
Der Gradient der Helligkeitsfunktion ergibt sich dann aus
  \[ \nabla I(x,y) = \sqrt{(I \circ K_X)(x,y)^2 + (I \circ K_Y)(x,y)^2} \]

\subsection*{Sobel-Operator}
Eine der häufigsten Faltungsmatrizen ist der \textsc{Sobel}-Operator.
Dieser funktioniert für unsere Anwendung recht gut, ist jedoch anfällig für Bildrauschen.
  \[ K_X = \begin{vmatrix}
       -1 & -2 & -1 \\
        0 &  0 &  0 \\
        1 &  2 &  1
     \end{vmatrix}
     \quad
     K_Y = K_X^\top = \begin{vmatrix}
       -1 & 0 & 1 \\
       -2 & 0 & 2 \\
       -1 & 0 & 1
     \end{vmatrix} \]

\subsection*{Roberts-Operator}
Der \textsc{Roberts}-Operator \cite{DBLP:books/garland/Roberts63} war eine der ersten Faltungsmatrizen die zur Kantenerkennung in technischen Strichzeichnungen verwendet wurde.
Da die Matrix $K_X$ identisch mit ihrer Transponierten ist verwendet man für die zweite Dimension eine Matrix $K_Y$ deren Spalten vertauscht wurden.
  \[ K_X = \begin{vmatrix}
       1 &  0 \\
       0 & -1
     \end{vmatrix}
     \quad
     K_Y = \begin{vmatrix}
        0 & 1 \\
       -1 & 0
     \end{vmatrix} \]
