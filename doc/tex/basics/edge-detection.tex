\section*{Kantenerkennung}

Kanten in einem Bild zeigen sich durch abrupte Helligkeitsänderungen zwischen
benachbarten Pixeln.
Um die Helligkeitsänderung für jedes Pixel zu bestimmen muss der Gradient der
Helligkeitsfunktion eines Bildes berechnet werden.
Das gängigste Verfahren zur Gradientenbestimmung ist die Faltung des Bildes $I$
mit einer Faltungsmatrix $K$.
  \[ (I \circ K)(x) = \sum_{t\in\mathcal{D}^*(K)} I(x+t) K(t)
       \quad \forall x\in\mathcal{D}(I) \]
Eine der häufigsten Faltungsmatrizen ist der \textsc{Sobel}-Operator:
  \[ K_{Sobel} =  \begin{vmatrix}
           -1 & -2 & -1 \\
            0 &  0 &  0 \\
            1 &  2 &  1
         \end{vmatrix} \]
Dieser funktioniert für unsere Anwendung recht gut. Erfasst aber bereits leichte
Helligkeitsänderungen.
Daher haben wir auch andere Matrizen ausprobiert.
Unter anderem den \textsc{Roberts}-Operator \cite{DBLP:books/garland/Roberts63}.
  \[ K_{Roberts} = \begin{vmatrix}
           -1 & 0 \\
            0 & 1
         \end{vmatrix} \]
Dies ist einer der ersten Operatoren der zur Kantenerkennung in
Strichzeichnungen entwickelt wurde.

Die Berechnung eines einzelnen Gradientenbildes reicht jedoch nicht aus um alle
Kanten eines Bildes zu erfassen, da die Faltungsmatrix nur Helligkeitsänderungen
entlang einer Dimension erfasst.
Deshalb muss ein zweites Gradientenbild unter Verwendung der Transponierten der
Faltungsmatrix berechnet werden.
Das Kantenbild ergibt sich dann aus der Kombinitation beider Gradientenbilder
mittels pixelweiser Addition der Absolutwerte.
  \[ I_{edge} = \Big|I \circ K\Big| + \Big|I \circ K^\top\Big| \]
