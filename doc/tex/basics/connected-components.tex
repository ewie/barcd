\section*{Connected Components Labeling}

Um zusammenhängende Bildstrukturen zu erfassen verwendet man dass sogenannte
Connected Components Labeling. Dabei wird jeder Bildpunkt genau einer Menge
zugeordnet welche je eine Komponente zusammenhängenger Bildpunkte repräsentiert.
Zwei Bildpunkte hängen zusammen, wenn diese benachbart sind und gleichen Wert
besitzen. Für die Nachbarschaft gibt es verschiedene Modelle.

\begin{description}
  \item[4-Konnektivität]
    betrachtet für ein Pixel $(x,y)$ die 4 Pixel $(x\pm1,y)$ und $(x,y\pm1)$
  \item[8-Konnektivität]
    betrachtet für ein Pixel $(x,y)$ neben den 4 Pixel der 4-Konnektivität
    zusätzlich noch die 4 Pixel $(x\pm1,y\pm1)$
\end{description}

Je mehr Pixel als Nachbarschaft herangezogen werden, desto größer können die
zusammenhängenden Komponenten ausfallen, da es mehr Möglichkeiten gibt in denen
zwei Pixel als verbunden gelten.
