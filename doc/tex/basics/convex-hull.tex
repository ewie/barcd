\section*{Konvexe Hülle}

Als konvexe Hülle einer Punktmenge $M$ bezeichnet man das minimale konvexe
Polygon, welches sämtliche Punkte aus $M$ enthält.
Für die Berechnung der konvexen Hülle aus einer Menge von Punkten existieren
diverse Alogirthmen.
Ein simpler Algorithmus ist Andrew's Monotone Chain \cite{compgeom2008}.
Dazu werden zwei "`halbe"' Hüllen erstellt, die obere Hülle $U$ und untere Hülle
$L$.
Die obere Hülle wird erzeugt indem über alle Punkte der Menge $M$ entsprechend
der lexikografischen Ordnung
  \[ (x,y) < (u, v) \Longleftrightarrow x < u \vee x = u \wedge y < v \]
iteriert, und wiederholt das letzte Element in $U$ entfernt wird solange $U$
mindestens 2 Punkte enthält und die letzten 2 Punkte zusammen mit dem aktuell
betrachteten Punkt \emph{keine} Rechtskurve bilden.
Anschließend wird der aktuelle Punkt ans Ende von $U$ eingefügt.
Die untere Hülle $L$ wird analog zu $U$ erzeugt.
Jedoch werden die Punkte in umgekehrter Reihenfolge betrachtet.
Zudem \emph{müssen} die letzten 2 Punkte zusammen mit dem aktuell betrachteten
Punkt eine Rechtskurve bilden. Abschließend werden $U$ und $L$ so zusammen
gefasst, dass die konvexe Hülle als Liste von Eckpunkten im Uhrzeigersinn
ergibt.
