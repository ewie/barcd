\subsubsection*{Dilation}

Die Dilatation ist eine morphologische Operation um Strukturen eines Bildes zu
vergrößern.
Für die Dilation eines Grauwertbildes wird jedem Pixel der Maximalwert aller
Pixel innerhalb einer gewissen Umgebung zugewiesen.
Für die Beschreibung der Umgebung verwendet man ein Strukturelement fester
Größe und Form (z.B. ein 3$\times$3 Quadrat).

  \[ (I \oplus K)(x) =
       \max \{ I_{x+t} | t \in D(K) \} \forall x \in D(I) \]

Durch die Anwendung einer Dilation können Bildstrukturen zusammengefasst
werden wenn deren Abstand kleiner ist als die Größe des Strukturelements.
