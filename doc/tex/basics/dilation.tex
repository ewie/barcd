\section{Dilation}
\writtenby{\dcauthornameewie}%
% Erosion und Dilation, Öffnen und Schliessen
% aus Voss & Süße 1991
Die Dilatation ist eine morphologische Operation um Strukturen eines Bildes zu vergrößern.
Bei die Dilation eines Grauwertbildes wird jedem Pixel der Maximalwert aller Pixel innerhalb einer gewissen Umgebung zugewiesen~\cite[Kapitel~3.5]{steinmueller2008}.
Zur Beschreibung der Umgebung dient ein Strukturelement~$X$ fester Größe und Form~(z.B. ein 3$\times$3-Quadrat), welches mit seinem Mittelpunkt~$(0,0)$ über dem betrachteten Pixel platziert wird.
\begin{equation}
  (I\oplus X)(x,y)= \max \left\{ I(x+u,y+v) \;\middle|\; (u,v) \in X \right\} \quad X \subseteq \mathbb{Z}^2
\end{equation}
Eine Dilation findet Anwendung um Bildstrukturen zu verstärken und zusammenzufassen~\cite[Kapitel~3.1]{steinmueller2008}.
Es können aber nur Bildstrukturen zusammengefasst werden sofern das Strukturelement groß genug ist um Zwischenräume zu überbrücken.

\begin{figure}[H]
  \label{fig:dilation}
  \centering
  \begin{subfigure}[t]{.49\linewidth}
    \centering
    \includegraphics[width=0.4\textwidth]{img/basics/dilation/before}
    \caption{Ausgangsbild (100$\times$100 Pixel)}
  \end{subfigure}
  \begin{subfigure}[t]{.49\linewidth}
    \centering
    \includegraphics[width=0.4\textwidth]{img/basics/dilation/after}
    \caption{Dilation mit einem 3$\times$3-Strukturelement}
  \end{subfigure}
  \caption{Beispiel einer Dilation eines Grauwertbildes}
\end{figure}

