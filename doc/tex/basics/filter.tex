\section{Filter}
\writtenby{\dcauthornameriren}%
Für viele Bildverarbeitungsprozesse werden Filter genutzt, die sich z.B. der mathematischen Faltung bedienen, um Rauschen zu entfernen, Kanten zu schärfen oder auch Kanten zu erkennen. Die zur Bildverbesserung benutzen Filter, werden hier kurz erläutert.

\subsection*{Pixelbasierte Bildverbesserung}
Bei der pixelbasierten Bildverbesserung wird jeder Pixelfarbwert einzeln berechnet, ohne auf die Umgebung dieses Pixels Rücksicht zu nehmen. Die so neu entstandenen Werte ersetzen dann die alten Farbwerte des Pixels.
So kann das Bild z.B. aufgehellt und der Kontrast verstärkt werden.
\begin{figure}[H]
  \centering
  \includegraphics[height=4cm]{img/QR/perfect_real_01.jpg}
  \includegraphics[height=4cm]{img/QR/qr-bright.jpg}
  \caption{Pixelbasierte Bildverbesserung.}
  \label{fig:pixelbright}
\end{figure}
%Quelle Tönnies2005

\subsection*{Faltungsfilter}
Faltungsfilter, auch lineare Filter genannt, machen sich das Prinzip der Faltung zu nutzen, dass bereits im Abschnitt der 'Kantenerkennung' beschrieben wurde. 

Im Fall der Bildbearbeitung gibt es weitere Anwendungsmöglichkeiten für entsprechende Faltungsmatritzen, z.B. für die Schärfung des Bildes oder der Ausgleich von Rauschen.
Für den Rauschausgleich gibt es eine Spezialform dieser Faltungsfilter, die Gauß-filter genannt wird, und bei der die Faltungsmatrix den Werten einer Gaußglocke, mit dem Maximum in der Mitte der Matrix, entspricht. Dadurch erreicht man eine ausgleichende Wirkung auf verrauschte Bilder, denn alle Pixel werden an ihre umgebenden Pixel angepasst, was aber zusätzlich den Effekt hat, dass das Bild unschärfer wird.

\begin{equation}
  K_{sharpen} = \begin{vmatrix}
    -1 & -1 & -1 \\
    -1 &  9 & -1 \\
    -1 & -1 & -1
  \end{vmatrix}
  \quad
  K_{Gauss} = \begin{vmatrix}
    \frac{1}{16} & \frac{1}{8} & \frac{1}{16} \\
    \frac{1}{ 8} & \frac{1}{4} & \frac{1}{ 8} \\
    \frac{1}{16} & \frac{1}{8} & \frac{1}{16}
  \end{vmatrix}
\end{equation}
\begin{figure}[H]
  \centering
  \includegraphics[height=4cm]{img/QR/perfect_03.jpg}
  \includegraphics[height=4cm]{img/QR/qr-gauss.jpg}
  \includegraphics[height=4cm]{img/QR/qr-sharp.jpg}
  \caption{Original, gauß- und danach wieder schärfe-gefiltertes Bild}
  \label{fig:sharpgauss}
\end{figure}
%Quelle Tönnies2005

\subsection*{Unscharfes Maskieren}
\label{sec:unsharp}
%sharpenimage
Beim unscharfen Maskieren, wird ein Originalbild verdreifacht und eine Version davon mit z.B. einem Gauß-Filter unscharf gemacht. Danach wird dieses unscharfe Bild von der zweiten Version abgezogen, insofern ein Schwellwert überschritten ist, und so ein Bild mit hervorgehobenen Kanten erhalten. Dieses wird nun zum dritten Bild addiert und in diesem dadurch die Kanten verstärkt, was einer Schärfung des Originalbildes entspricht.
\begin{figure}[H]
  \centering
  \includegraphics[height=4cm]{img/QR/blurry_03_3.jpg}
  \includegraphics[height=4cm]{img/QR/qr-unsharpmask.jpg}
  \includegraphics[height=4cm]{img/QR/qr-unsharpmask-sharp.jpg}
  \caption{Unscharfes Maskieren mit Gauß (Original, Maske, Ergebnis)}
  \label{fig:unsharpmask}
\end{figure}


\subsection*{Laplace-Operator}
%isBlurry
Den Verlauf von Kanten kann man auch durch die zweite Ableitung über den Pixelwerten des Bildes erhalten. Denn dort wo diese von positiven Werten zu negativen wechselt, hat die erste Ableitung ein Maximum und somit ist an dieser Stelle ein sehr starker Helligkeitsunterschied im Bild vorhanden, der auf eine Kante hindeutet. Dieser Vorzeichenwechsel wird auch als Nulldurchgang bezeichnet. Berechnet wird der Laplace-Operator durch:
\begin{equation}
  \nabla^2 f(x,y) = \frac{\partial^2 f}{\partial x^2}(x,y) + \frac{\partial^2 f}{\partial y^2}(x,y) + \frac{\partial^2 f}{\partial x \partial y}(x,y) + \frac{\partial^2 f}{\partial y \partial x}(x,y)
\end{equation}
Für schnellere Berechnungen können aber auch 2 angenäherte Faltungsmatrizen genutzt werden:
\begin{equation}
  K_4 = \begin{vmatrix}
     0 & -1 &  0 \\
    -1 &  4 & -1 \\
     0 & -1 &  0
  \end{vmatrix}
  \quad
  K_8 = \begin{vmatrix}
    -1 & -1 & -1 \\
    -1 &  8 & -1 \\
    -1 & -1 & -1
  \end{vmatrix}
\end{equation}
\begin{figure}[H]
  \centering
  \includegraphics[height=4cm]{img/QR/perfect_03.jpg}
  \includegraphics[height=4cm]{img/QR/qr-laplace.jpg}
  \caption{Laplace-Filter($K_8$) mit Original}
  \label{fig:sharpgauss}
\end{figure}
%Quelle Tönnies2005 S 182



\subsection*{Laplacian-of-Gaussian-Filter}
\label{sec:LoG}
%isBlurry
Der Laplace Operator allein ist durch die Reaktion auf jeden stärkeren Helligkeitsunterschied sehr empfindlich für Rauschen und wird deswegen auch zusammen mit einem Gauß-Filter verwendet. Durch diese Kombination wird das Rauschen abgeschwächt, aber die Reaktivität auf Kanten bleibt erhalten. Dieser Filter wird auch Marr-Hildreth-Filter oder Mexican-Hat-Filter(wegen des Aussehens des Filters) genannt. Für die Berechnung werden die zweiten Ableitungen der Gaußfunktion benötigt:
\begin{equation}
  LoG(x,y) = -\frac{1}{\pi \sigma^4}(1-\frac{x^2 + y^2}{2\sigma^2}) exp(-\frac{x^2 + y^2}{2\sigma^2})
\end{equation}
Diesen Filter kann man sich noch auf andere Weise zu nutzen machen, denn die positiven Werte des Laplace-Operators repräsentieren, wie stark die erste Ableitung ansteigt und damit wie stark die Veränderungen von Pixelpaar zu Pixelpaar ist. Somit ergibt sich daraus ein Wert für die Unschärfe des Bildes, der maximale Wert im aktuellen Bild beschreibt diese zum Beispiel. Um so größer er ist, um so schärfer ist das Bild und wenn er kleiner ist, ist auch das Bild unschärfer.
\begin{figure}[H]
  \centering
  \includegraphics[height=4cm]{img/QR/perfect_03.jpg}
  \includegraphics[height=4cm]{img/QR/qr-LoG.jpg}
  \caption{Laplacian-of-Gaussian-Filter mit Original}
  \label{fig:sharpgauss}
\end{figure}
%Quelle Tönnies2005 S 183