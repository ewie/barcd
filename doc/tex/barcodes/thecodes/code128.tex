\subsection*{Code128}
%Herkunft und Nutzung
Der Code 128 wurde 1981 von Ted Williams(Computer Identics) vorgestellt. Er kodiert den kompletten ASCII 128 Zeichensatz und wird viel in der Gesundheitsindustrie, bei Blutbanken und der Elektronikherstellung genutzt.
%Aufbau

%Vorteile/Nachteile und zusätzliches
a more compact and flexible code for warehouse and distribution applications thanks to its continuous, self-checking bidirectional features
\begin{itemize}
	\item Startsymbol, Nutzinformation, Prüfsymbol und Stoppsymbol
	\item Mehrbreitencode mit elf Modulen für jedes Symbol
	\item drei Balken und drei Zwischenräume von maximal vier Modulen Breite bilden das Codemuster für ein Symbol
	\item Überprüfung: Anzahl als Balken dargestellter Module stets gerade, Anzahl der Zwischenraummodule ungerade
	\item \url{http://www.tec-it.com/de/support/knowbase/barcoding/untiltoday/Default.aspx}
	\item \url{http://www.barcode-labels.com/technical-support/techy-tip-code-128}
\end{itemize}
