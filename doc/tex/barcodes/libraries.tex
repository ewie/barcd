\section{Strichcode Bibliotheken}
\writtenby{\dcauthornameriren}%
Im Internet existieren viele verschiedene Bibliotheken zur Generierung und Dekodierung von Strichcodes aller Arten.

Hier werden einige aufgeführt und ZXing, die im Programm genutzte, wird etwas näher erläutert.

Außerdem wird darauf eingegangen, warum ZXing genutzt wird.


\subsection*{Open-Source Barcode Generatoren}

\paragraph*{Barcode Writer}
Der Barcode Writer ist in PostScript geschrieben und kann angeblich alle Formate erstellen. Er untersteht der 'MIT/X-Consortium License'. Man findet ihn unter: \url{https://code.google.com/p/postscriptbarcode}

\paragraph*{Barcode4J}
Barcode4J ist in Java geschrieben und kann sehr viele Strichcodes generieren. Barcode4J untersteht der 'Apache Licence v2.0'. Genaueres erfährt man hier: \url{http://barcode4j.sourceforge.net/index.html}

\paragraph*{www.barcoding.com}
Unter \url{http://www.barcoding.com/upc} kann man viele 1D Codes online erstellen und als jpg-Dateien herunterladen.

\paragraph*{www.terryburton.co.uk}
Unter \url{http://www.terryburton.co.uk/barcodewriter/generator} kann man online jegliche Art von Strichcode generieren und in verschiedenen Dateiformaten (jpg, png, eps) herunterladen.


\subsection*{Open-source Barcode Scanner}

\paragraph*{ZBar}
ZBar ist betriebssystemunabhängig in C geschrieben, besitzt aber Schnittstellen zu C++, Perl und Python. Es erkennt mindestens die wichtigsten Strichcodes und steht unter der 'GNU LGPL 2.1'. Mehr Informationen findet man unter \url{http://zbar.sourceforge.net}.

\paragraph*{BarBara Barcode Library}
BarBara ist ein Strichcodescanner in VB6, VB.NET und PHP5. Viele Strichcodes werden unterstützt und die Bibliothek ist unter der 'GNU LGPL' verfügbar. Weitere Informationen kann man hier finden: \url{http://sourceforge.net/projects/barbara}


\subsection*{ZXing('Zebra Crossing')}
ZXing ist eine Open-Source Bibliothek in Java zum lesen von vielen unterschiedlichen 1D/2D Strichcodes. Sie ist eigentlich auf Codeerkennung auf Handys spezialisiert, kann aber ohne Probleme auch als Dekodierungsbibliothek genutzt werden.

Einige der unterstützten Codes sind: UPC-A/UPC-E, EAN-8/EAN-13, Code 39, Code 128, ITF, Codabar, RSS-14(all variants), QR Code und Data Matrix.

ZXing wird unter der Apache License v2.0 zur Verfügung gestellt und weitere Informationen kann man unter folgender Adresse finden: \url{http://code.google.com/p/zxing}

Die Entscheidung für ZXing ist vor allem dadurch zu begründen, dass es bereits in Java geschrieben ist und jeglichen Strichcode erkennt. Darüber hinaus ist die Bibliothek auch einfach einzubinden.
