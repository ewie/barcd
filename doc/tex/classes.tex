\chapter{Implementierung}
\writtenby{\dcauthornameriren}%
Dieses Kapitel widmet sich der Beschreibung der wesentlichen Aspekte der Implementierung.
Dazu ist nicht jede einzelne Funktion aufgeführt, sondern nur die wichtigsten, die für das Verständnis der Verarbeitung erforderlich sind.
Am Ende des Kapitels werden zudem die genutzten Bibliotheken beschrieben.

Das hier abgebildete Diagramm zeigt die Beziehungen zwischen den zentralen Komponenten der Implementierung auf die im Folgenden einzeln eingegangen wird.
Aus Platzgründen sind die Klassendiagramme nur auf die öffentlichen sowie geschützten Methoden beschränkt.
Klassen, die lediglich eine Abhängigkeit darstellen und nicht Bestandteil einer Bibliothek sind, werden nur mit ihrem Namen dargestellt.
\elasticfigure{img/uml/uml.0}{zentrale Komponenten der Implementierung und deren Beziehungen untereinander}

\section{Klasse \code{Job}}
Die Klasse \code{Job} kapselt die Bildquelle~(Methode \code{getSource}) und erzeugt fortlaufend nummerierte Extractions~(Methode \code{createExtraction}).
Zudem stellt es für die Extractions ein URL Template~(Methode \code{getExtractionUrlTemplate}) zur Verfügung um jeder Extraction eine URL zuzuordnen die zur Speicherung der Extraction nötig ist.
Durch die Initialisierung eines Jobs mit einer optionalen Framenummer~(Parameter \code{initialFrameNumber}) ist es möglich die Verarbeitung eines Jobs an einer gewünschten Stelle wieder fortzusetzen.

\elasticfigure{img/uml/uml.115}{Klassendiagramm von \code{Job}}


\section{Klasse \code{Extraction}}
Eine Instanz der Klasse \code{Extraction} kapselt alle Regionen~(\code{getRegions()}) und Barcodes~(\code{getBarcodes()}) die aus genau einem Bild extrahiert wurden.
Jede Instanz trägt, beginnend bei 0, die Nummer des entsprechenden Frames des Quellmaterials~(\code{getFrameNumber()}).
Die Barcodes einer Extraction sind in zwei Kategorien unterteilt.
\begin{itemize}
  \item gebundene Barcodes~(\code{getRegionBarcodes()}), die von einer Region erfasst sind
  \item ungebundene Barcodes~(\code{getRegionlessBarcodes()}), die von keiner Region erfasst sind
\end{itemize}

\elasticfigure{img/uml/uml.110}{Klassendiagramm von \code{Extraction}}


\section{Klasse \code{Region}}
Instanzen der Klasse \code{Region} beschreiben Bildregionen.
Als Grundlagen dienen z.B. die zusammenhängenden Komponenten eines Bild.
Eine Region ist durch ein konvexes Polygon~(Methode \code{getConvexPolygon}) und einen Deckungsgrad~(Methode \code{getCoverage}) beschrieben.
Die Methode \code{getOrientedRectangle} erzeugt aus dem konvexen Polygon das umschließende Rechteck mit minimaler Fläche.
Im Gegensatz dazu erzeugt die Methode \code{getAxisAlignedRectangle} das achsenparallele Rechteck.

\elasticfigure{img/uml/uml.116}{Klassendiagramm von \code{Region}}


\section{Klasse \code{Barcode}}
Die Klasse \code{Barcode} dient der Beschriebung erfolgreich gelesener Barcodes.
Ein Barcode besteht aus seinem Typ~(\code{getType()}), dem kodierten Text~(\code{getText()}), den Binärdaten~(\code{getBytes()}) sowie mehreren Punkten~(\code{getAnchorPoints()}), die zum Lesen des Barcodes dienten.
Die Punkte können z.B. der Start- und Endpunkt der Scanline für einen eindimensionalen Barcode sein oder die Mittelpunkte der Marker bei einem zweidimensionalen Barcode~(z.B. QR Code).

\elasticfigure{img/uml/uml.101}{Klassendiagramm von \code{Barcode}}


\section{Klasse \code{Source}}
Das Klasse \code{Source} stellt eine Bildquelle dar, die ein Objekt erzeugen kann welches Bilder bereitstellt~(\code{createImageProvider()}).
Eine konkrete Bildquelle kapselt dazu die Informationen die zur Erzeugung einer Bildquelle nötig sind (z.B. die URL einer Videoresource).

\elasticfigure{img/uml/uml.119}{Klassendiagramm von \code{Source}}

\subsection*{Konkrete Implementierungen}
Insgesamt existieren 6 Realisierungen von \code{Source} die sich im Laufe der Entwicklung als sinnvoll erwiesen haben.
Die Klasse \code{SourceFactory} stellt Factory-Methoden zu Verfügung um die hier gelisteten Implementierungen zu erzeugen.

{
\setlength{\leftmargini}{1.5em}
\setlength{\labelsep}{\textwidth}
\begin{description}
  \item[\code{BufferedImageSource}]
    Verwendet eine geordnete Sammlung von \code{BufferedImage}-Instanzen.
    Diese Klasse ist sinnvoll wenn die Bilder bereits von einer Anwendung geladen sind.
  \item[\code{ImageCollectionSource}]
    Verwendet eine Liste von URLs um Bilder lokal oder aus einem Netzwerk (z.B. über HTTP) zu laden.
  \item[\code{ImageSequenceSource}]
    Verwendet eine URL um durchnummerierte Bilder (ebenfalls lokal oder aus einem Netzwerk) zu laden.
    Ein Anwendungsfall ist z.B. Bilder die zuvor bereits aus einem Video extrahiert wurden und daher durchnummeriert vorliegen.
  \item[\code{ImageSnapshotServiceSource}]
    Verwendet eine URL für einen Webservice, der bei jedem Request ein neues Bild liefert (z.B. eine Netzwerkwebcam).
  \item[\code{VideoDeviceSource}]
    Nutzt eine Videogerät, welches über eine Zahl, beginnend bei 0, identifiziert wird.
    Zum Lesen der Videodaten dient JavaCV\footnote{\url{http://code.google.com/p/javacv/}} um auf die C/C++-API von OpenCV\footnote{\url{http://opencv.org}} zugreifen zu können.
  \item[\code{VideoFileSource}]
    Verwendet eine URL zum laden eines Videos, welches lokal aber auch in einem Netzwerk (Zugriff über HTTP) vorliegen kann.
    Nutzt ebenfalls JavaCV zum Lesen der Daten.
\end{description}
}


\section{Klasse \code{Extractor}}
Der \code{Extractor} stellt die Implementierung des in \autoref{sec:extraction} beschriebenen Verfahrens.
Als Eingabe dient eine Instanz von \code{Job}.
Die Verarbeitung der Bilder erfolgt sequentiell durch wiederholten Aufruf der Methode \code{processNextImage}.
Dem, mit der Methode \code{setExtractionHandler} gesetzten, \code{ExtractionHandler} werden nach jedem verarbeiteten Bild die erzeugte \code{Extraction}-Instanz sowie das entsprechende Bild übergeben.

Die einzelnen Verarbeitungsschritte des Extractors sind über Interfaces realisiert, wodurch es möglich ist das Verfahren durch alternative Implementierungen anzupassen.
Das Grauwertbild wird mittels eines \code{Grayscaler}~(Methode \code{setGrayscaler}) bestimmt.
\code{RegionExtractor}~(Methode \code{setRegionExtractor}) ist für das Auffinden von Regionen zuständig.
Mit \code{Filter<Region>}~(Methode \code{setRegionFilter}) ist es möglich Regionen nach bestimmten Kriterien~(Größe, etc.) auszuwählen.
Für jede gewählte Region wird mit einem \code{ImageEnhancer}~(Methode \code{setImageEnhancer}) der entsprechende Bildausschnitt bildtechnisch aufbereitet.
Abschließend erfolgt unter Verwendung eines \code{BarcodeReader}~(Methode \code{setBarcodeReader}) das Lesen des Barcodes im Bildausschnitt einer jeden Region sowie im gesamten Bild.

\elasticfigure{img/uml/uml.109}{Klassendiagramm von \code{Extractor}}

\subsection{Interface \code{Grayscaler}}
Der \code{Grayscaler} berechnen das Grauwertbild eines Bildes.
Aus Performancegründen setzt die Implementierung \code{DefaultGrayscaler} keines der in \autoref{sec:grayscale} vorgestellten Verfahren um.
Stattdessen wird das Eingabebild in das, nur einen Farbkanal besitzenden, Ausgabebild gezeichnet\footnote{\href{http://docs.oracle.com/javase/6/docs/api/java/awt/Graphics2D.html\#drawImage(java.awt.image.BufferedImage, java.awt.image.BufferedImageOp, int, int)}{java.awt.Graphics2D\#drawImage()}}.

Eine Messung der Zeit, die zur Konvertierung von mehreren Bildern mit einer Größe von 1920$\times$1080 Pixel (Bildmaterial eines Full-HD-Videos) nötig ist, zeigte, dass eine Implementierung nach \autoref{sec:grayscaling} rund $100\unit{ms}$ pro Bild benötigt.
Wohingegen die Implementierung des \code{DefaultGrayscaler} mit einer Laufzeit von etwa $0\unit{ms}$ pro Bild nicht messbar war.
\elasticfigure{img/uml/uml.148}{Klassendiagramm von \code{DefaultGrayscaler}}


\subsection*{Interface \code{RegionExtractor}}
Implementationen von \code{RegionExtractor} dienen dem Auffinden von Regionen in einem Bild.
Mit \code{DefaultRegionExtractor} ist das in \autoref{sec:candidate-extraction} beschriebene Verfahren realisiert 
\elasticfigure{img/uml/uml.146}{Klassendiagramm von \code{DefaultRegionExtractor}}


\subsection{Interface \code{ImageEnhancer}}
Mit einem \code{ImageEnhancer} werden die Bildausschnitte aufbereitet um die Chancen zu erhöhen, dass der \code{BarcodeReader} einen Barcode erfolgreich lesen kann.
Die Standardimplementierung ist mit \code{DefaultImageEnhancer} gegeben.
\elasticfigure{img/uml/uml.149}{Klassendiagramm von \code{DefaultImageEnhancer}}


\subsection*{Interface \code{BarcodeReader}}
Die Funktion eines \code{BarcodeReader} liegt im Lesen von Barcodes in einem Bild.
Die Implementierung \code{DefaultBarcodeReader} verwendet ZXing~(siehe \autoref{par:zxing}) zum Dekodieren der Barcodes.
%\elasticfigure{img/uml/uml.147}{Klassendiagramm von \code{DefaultBarcodeReader}}


\begin{figure}[h]
  \centering
  \elasticgraphic{img/uml/uml.147}
  \caption{Klassendiagramm von \code{DefaultBarcodeReader}}
\end{figure}




\section{Serialisierung}
Um Jobs und Extractions speichern und zwischen Anwendungen austauschen zu können ist eine Serialisierung, d.h. die Übersetzung von Objekten in eine maschinenlesbare Form sowie die Wiederherstellung jener Objekte, erforderlich.

XML stellt für diesen Zweck die ideale Lösung dar, da es ein weit verbreiteter Standard zur externen Repräsentierung von Daten darstellt.
Zudem wird an der Professur Medieninformatik im Bereich Informationretrieval bereits die XML-Datenbank eXist\footnote{\url{http://www.exist-db.org/}} eingesetzt wodurch eine XML-Lösung naheliegend erscheint.
Zusammen mit einem XML Schema (siehe \autoref{annex:xml-schema}) ist es möglich die Serialisierung weitesgehend zu automatisieren, da Werkzeuge existieren (z.B. JAXB für Java) die aus einem XML Schema Quellcode erzeugen, der die Serialisierung realisiert.

Die Grundlage der XML-Serialisierung bildet die abstrakte Klasse \code{XmlSerialzer}, die wiederum eine Realisierung des Interface \code{Serializer} ist.
Konkrete Serialisierer wie z.B. \code{XmlJobSerializer} für \code{Job} und \code{XmlExtractionSerializer} für \code{Extraction} müssen ein Objekt des entsprechenden Typs auf ein XML-Element abbilden (Methode \code{createRootElement}) bzw. aus einem XML-Element ein Objekt wieder herstellen (Methode \code{restoreModel}).
Neben Zeichenketten und Bytestreams als Ein- und Ausgabe bietet \code{XmlSerializer} zusätzlich DOM-Knoten als Quelle und Ausgabe.

Optional kann \code{XmlSerializer} die Eingabe bzw. Ausgabe validieren, indem mit der Methode \code{setSchema} das XML Schema gesetzt wird.
Das Schema kann auch über eine URL mittels der Methode \code{setSchemaLocation} verwendet werden.
Ist die URL des Schemas bekannt, so kann diese bei der Serialisierung mit ausgegeben\footnote{\url{http://www.w3.org/TR/xmlschema-1/\#xsi_schemaLocation}} werden.
Die Methode \code{setUrlContext} setzt den URL-Kontext um relative URLs (z.B. relative Dateipfade) auflösen zu können.

\elasticfigure{img/uml/uml.150}{Klassendiagram von \code{XmlSerialzer}}



\section{Klasse ImageProcessing}
\writtenby{\dcauthornameriren}%
Das Ziel dieser Klasse ist es, einzelne Eigenschaften von Strichcodebilder so aufzuwerten, dass der enthaltene Strichode besser von der Bibliothek ZXing erkannt werden kann.
Zusätzlich sind auch noch einige weitere Funktionen implementiert, die teilweise andere Varianten oder einfach zusätzliche Funktionen darstellen. Auf diese wird hier aber nicht näher eingegangen, da sie keinen direkten Bezug zur aktuellen Implementation haben. Sie wurden aber nicht aus der Klasse entfernt, um die mögliche spätere Weiterentwicklung zu erleichtern.

Die Bildverarbeitung erfolgt mittels ImageJ\footnote{\url{http://rsbweb.nih.gov/ij/}}, ein gemeinfreies (public domain) Bildbearbeitungsprogramm, dessen Bibliotheken ebenfalls frei nutzbar sind.

\elasticfigure{img/uml/uml.127}{Klassendiagramm von \code{ImageProcessing}}

\subsection*{Methode hasShade}
Diese Methode erhält ein Eingabebild, überprüft ob es Schatten enthält und gibt einen Wahrheitswert als Antwort zurück.
Die möglichen Schatten werden über den weißen Rand der Strichcodes gesucht. Wenn dort ein ausreichend großer Helligkeitsunterschied auftritt, wird angenommen, dass es einen Schatten gibt, der nicht mehr von ZXing erkannt werden kann.
Um die Erkennung möglichst variabel zu gestalten, wurde zusätzlich ein Faktor für die Genauigkeit der Erkennung eingeführt. Er bestimmt die Anzahl an Pixeln, die, pro Seite, in die Untersuchung aufgenommen werden.
Die Voraussetzung für die korrekte Funktionsweise der Methode ist, dass die Ruhezone, um den Strichcode, nicht überlagert wird. Die Schattenerkennung wäre durch andere Objekte in dieser Zone korrumpiert, da durch diese definitiv die Helligkeitsunterschiede in diesem Bereich zu stark beeinflusst würden.


\subsection*{Methode isBlurry}
Diese Methode erhält ein Eingabebild und gibt einen Grad an Unschärfe für dieses zurück.
Das Bild wird dafür mit einem Laplacian-of-Gaußian-Filter bearbeitet und dadurch wird ein Bild erhalten, indem, wie in Kapitel \ref*{sec:LoG} beschrieben, der größte Wert die Schärfe der Abbildung repräsentiert.
Der Ausgabewert ist allerdings abhängig von der allgemeinen Helligkeit des Bildes, daher sollte ein bereits normalisiertes Bild als Eingabe dienen.


\subsection*{Methode isDark}
Diese Methode erhält ein Eingabebild und errechnet die ungefähre Helligkeit des Eingabebildes.
Die Helligkeit wird durch ein Histogramm bestimmt, indem aus ihm der Durchschnittswert (wo die Hälfte aller Pixel heller bzw. dunkler sind) berechnet wird. Dabei entsprechen kleinere Werte mehr dunklen Pixeln.


\subsection*{Methode isRotated}
Diese Methode erhält ein Eingabebild und gibt den Rotationswinkel des enthaltenen Strichcodes zurück.
Um den Winkel zu bestimmten, werden vom linken Rand aus starke Helligkeitsunterschiede gesucht und diese gefundenen Punkte werden, von oben nach unten, in einem Array gespeichert. Danach wird überprüft, ob eine Gerade, durch zwei hintereinander liegende Punkte, auch relativ nah an einem dritten, darauf folgenden Punkt vorbei geht. Wenn ja, dann werden weitere Punkte mit dieser Geraden überprüft und so die Anzahl an Punkten auf ihr bestimmt. Die Gerade mit der maximalen Anzahl an Punkten sollte theoretisch dem Anfangs- oder Endstrich des Codes entsprechen.
Diese Methode ist zwar für 1D Strichcodes konzipiert, funktioniert aber nichtsdestotrotz auch für 2D Codes, da immer eine längere Linie am Rand gefunden werden kann.
Dieser Winkelbestimmung ist nicht exakt, aber reicht für die Erkennung durch die Bibliothek ZXing vollkommen aus, da Abweichungen von maximal 2$ \circ $ nicht ins Gewicht fallen.
 

\subsection*{Methode brightenBufferedImage\_linear}
Diese Methode erhält ein Eingabebild und einen Aufhellungsfaktor, der die Verstärkung der Beleuchtung angibt. Dadurch wird eine hellere bzw. dunklere Variante vom Bild zurückgegeben.
Zum Aufhellen oder entsprechenden Verdunkeln wird ein 'RescaleOp'-Filter genutzt, der entsprechend der pixelbasierten Bildverbesserung funktioniert. Durch ihn wird das Bild vom Normalzustand (Helligkeit=1.0) in einen aufgehellten (Helligkeit>1.0) bzw. abgedunkelten (Helligkeit<1.0) Zustand überführt (plus einem Offset von 15).
Dadurch entsteht eine gleichmäßige Aufhellung des ganzen Bildes.
Eine zweite, realistischere, Variante kann mit einem 'LookupTable'-Filter erzeugt werden, durch den das Bild vom Normalzustand durch eine Helligkeitszuweisung
($newPixelColor = ((oldPixelColor/255.0) * 255.0)^2$)zu jedem Pixel aufgehellt wird.
Dadurch entsteht eine realistische Aufhellung, da dunkle Pixel schneller hell werden. Dies entspricht aber nicht den Anforderungen, da die dunklen Teile für die Erkennung am wichtigsten sind. (Diese Funktion ist trotzdem für mögliche spätere Nutzung als brightenBufferedImage\_quadratic implementiert)


\subsection*{Methode findShades}
Diese Methode erhält ein Eingabebild und gibt das von möglichst allen Schatten befreite Bild zurück.
Die grundsätzliche Idee ist, die Kanten von Schatten zu finden und dann entsprechend den eingerahmten Bereich entsprechend aufzuhellen.
Nach der Implementierung ist allerdings klar geworden, dass der 'BackgroundSubtracter' von ImageJ dies bereits sehr gut tut.
Es wurde auch der Canny-Edge-Detector probiert, allerdings werden dabei zu viele mögliche Kanten erkannt. Also müsste für jedes Bild ein Schwellwert eingestellt werden, wodurch er sich als nutzlos erwies.


%\subsection*{Methode interpolateBufferedImage}
%Diese Methode erhält ein 'BufferedImage' und hellt einen viereckigen Bereich im Bild auf, welcher dann in einem anderen 'BufferedImage' zurückgegeben wird.
%
%Hellt einen Vier-/(Viel-)eckigen Teilbereich auf


\subsection*{Methode rotateBufferedImage}
Diese Methode erhält ein Eingabebild und gibt eine, um den angegebenen Winkel, gedrehte Variante davon zurück.
Das Bild wird über eine affine Transformation gedreht und dann der darzustellende Bereich angepasst.


\subsection*{Methode sharpenBufferedImage}
Diese Methode erhält ein Eingabebild und gibt eine schärfere Variante davon zurück.
Für die Schärfung wird die Methode der unscharfen Maskierung genutzt, wie sie in Kapitel \ref*{sec:unsharp} beschrieben wird.

\section{Klasse ImproveImage}
\writtenby{\dcauthornameriren}%
Das Ziel dieser Klasse ist es, einzelne Eigenschaften von Strichcodebilder so aufzuwerten, dass der enthaltene Strichode besser von der Bibliothek ZXing erkannt werden kann.
Dabei werden die einzelnen Methoden aus der Klasse ImageProcessing zusammen genutzt, um eine optimale Verbesserung der Bilder zu erreichen.

\elasticfigure{img/uml/uml.126}{Klassendiagramm von \code{ImproveImage}}

\subsection*{Methode checkBrightness}
Diese Methode erhält ein Eingabebild, überprüft ob es zu dunkel ist und hellt es wenn nötig auf.
Durch die Methode 'isDark', aus ImageProcessing, wird die Helligkeit des Bildes bestimmt und wenn diese unter der Hälfte der maximalen Helligkeit ist, dann wird sie bis dorthin, durch 'brightenBufferedImage\_linear', aufgehellt.
Nach der Aufhellung werden einige, nur leicht unscharfe, Bilder bereits nutzbar und der Kontrast wird für die spätere Verarbeitung verstärkt.


\subsection*{Methode checkBlur}
Diese Methode erhält ein Eingabebild, überprüft ob es zu unscharf ist und schärft es wenn nötig.
Durch die Methode 'isBlurry', aus ImageProcessing, wird die Schärfe des Bildes bestimmt und wenn diese unter 200 ist, dann wird sie durch 'sharpenBufferedImage' geschärft.


\subsection*{Methode checkRotation}
Diese Methode erhält ein Eingabebild, überprüft wie stark der enthaltene Strichcode gedreht ist und rotiert ihn wieder in eine lesbare Position.
Durch die Methode 'isRotated', aus ImageProcessing, wird der Rotationswinkel des Strichcodes bestimmt und dann wird das Bild durch 'rotateBufferedImage' gedreht.


\subsection*{Methode checkShadow}
Diese Methode erhält ein Eingabebild, überprüft ob Schatten enthalten sind und wenn ja, entfernt diese aus dem Bild.
Durch die Methode 'hasShade', aus ImageProcessing, wird überprüft, ob das Bild Schatten enthält und entfernt sie, wenn nötig, durch 'findShades' aus dem Bild.
 

\subsection*{Methode checkImage}
Diese Methode erhält ein Eingabebild und verbessert es so gut es geht.
Dabei wird als erstes die Helligkeit überprüft ('checkBrightness'), da sie nur positive Auswirkungen auf die folgenden Untersuchungen hat.
Darauf folgt die Untersuchung auf Schatten~('checkShadow'), da diese schlecht auf Umgebungen um den Strichcode reagiert und durch die Rotation möglicherweise ein Rahmen entstehen kann.
Als nächstes wird das Bild rotiert ('checkRotation'), weil dies vor dem Schärfen passieren muss, da das Bild durch die Rotation immer etwas unschärfer wird.
Der letzte Punkt ist dann die Schärfung ('checkBlur').
\section{Externe Bibliotheken}
%In diesem Bereich werden die Bibliotheken kurz vorgestellt, deren Funktionen für die Strichcodeerkennung und -verbesserung wichtig sind.
Die Implementierung verwendet diverse Bilbiotheken, die im folgenden kurz aufgeführt sind.

\paragraph{ImageJ}
ImageJ\footnote{\url{http://rsbweb.nih.gov/ij/}} ist ein gemeinfreies (public domain) Bildbearbeitungsprogramm, dessen Bibliotheken ebenfalls frei nutzbar sind.
Viele der Bildverbesserungsfunktionen des Programmes greifen auf Funktionen aus ImageJ zurück.

\subsection*{ZXing}
Wie bereits in Kapitel \ref{par:zxing} beschrieben, ist diese Bibliothek für das Decodieren der Strichcodes verantwortlich.

%\paragraph{ZXing}
%ZXing ist bereits in \autoref{sec:barcodelibs} beschrieben, ist diese Bibliothek für das Lesen der Strichcodes verantwortlich.



