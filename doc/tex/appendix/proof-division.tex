\chapter{Beweis der ganzahligen Division durch Rechtsshift}
\label{proof-division}
\writtenby{\dcauthornameewie}%
\paragraph{Satz}
Der Rechtsshift um $k\in\mathbb{N}_0$ Bits einer natürliche Zahl $b$ ergibt den Quotienten der ganzahligen Division von $b$ mit $2^k$.
\begin{equation}
  b >\!\!> k = \frac{b-r}{2^k} \quad r < 2^k
\end{equation}

\paragraph{Definition}
Sei $b$ eine Binärzahl der Form
\begin{equation}
  \label{eq:bin}
  b = \sum_{i=0}^{n-1} b_i 2^i \quad b_i \in \{0,1\}
\end{equation}
und der Rechtsshift von $b$ um $k$ Bits sei wie folgt definiert (alle Bits werden um $k$ Stellen nach rechts verschoben, wodurch die Bits $b_0$ bis $b_{k-1}$ wegfallen)
\begin{equation}
  \label{eq:rsh}
  b >\!\!> k = \sum_{i=k}^{n-1} b_i 2^{i-k} \quad k \in \mathbb{N}_0
\end{equation}

\paragraph{Beweis}
\begin{align*}
  b >\!\!> k &\overset{!}{=} \frac{b-r}{2^k}  \\
    &= 2^{-k} b - 2^{-k} r \\
    &= 2^{-k} \sum_{i=0}^{n-1} b_i 2^i - 2^{-k} \sum_{i=0}^{k-1} r_i 2^i
      \tag*{Anwendung von \eqref{eq:bin} auf $b$ und $r$} \\
    &= \sum_{i=0}^{n-1} b_i 2^i 2^{-k} - \sum_{i=0}^{k-1} r_i 2^i 2^{-k} \\
    &= \sum_{i=0}^{n-1} b_i 2^{i-k} - \sum_{i=0}^{k-1} r_i 2^{i-k} \\
    &= \sum_{i=k}^{n-1} b_i 2^{i-k} + \underbrace{\sum_{i=0}^{k-1} b_i 2^{i-k} - \sum_{i=0}^{k-1} r_i 2^{i-k}}_{\text{Null, da}~r_i=b_i~\forall i} \\
    &= \sum_{i=k}^{n-1} b_i 2^{i-k}
      \tag*{entspricht der Definition des Rechtsshifts \eqref{eq:rsh}}
\end{align*}
\hfill \qedsymbol
