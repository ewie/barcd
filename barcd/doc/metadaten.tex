%======================================================================
%	Metadaten
%======================================================================
%	$Id$
%	Matthias Kupfer
%======================================================================

\newcommand{\dcsubject}{Teamorientierte Projektarbeit}
% z.B. (Diplom/Studien/Haus)arbeit, Praktikumsbericht, Studie, Beleg, 
% (Pro/Haupt/Ober)seminar, Seminar usw.
\newcommand{\dctitle}{Entwurf und Implementation von Verfahren zum Auffinden und Lesen von Strichcodes in Videodaten}
\newcommand{\dcsubtitle}{~} % Untertitel, falls erforderlich

\newcommand{\dcauthornameewie}{Erik Wienhold}
\newcommand{\dcauthornameriren}{René Richter}
\newcommand{\dcauthoremailewie}{ewie@hrz.tu-chemnitz.de}
\newcommand{\dcauthoremailriren}{riren@hrz.tu-chemnitz.de}
\newcommand{\dcauthors}{%
    René Richter \normalsize\texttt{riren@hrz.tu-chemnitz.de} \\
    Erik Wienhold \normalsize\texttt{ewie@hrz.tu-chemnitz.de} \\
}
\newcommand{\dcdate}{\today} 

\newcommand{\dcplace}{Chemnitz} % Ort, kann an der TU meist so bleiben
\newcommand{\dcuni}{Technische Universität \dcplace}
\newcommand{\dcdepart}{Informatik} % Fakultätsangabe
\newcommand{\dcprof}{Medieninformatik} % Angabe der Professur

\newcommand{\dcpruefer}{}% Prüfer der Arbeit
\newcommand{\dcadvisor}{Robert Manthey}% Betreuer der Arbeit

\newcommand{\dckeywords}{Liste,von,Stichworten,die,als,Schlagworte,geeignet,
sind}

%%======================================================================
% Einstellungen des Hyperref-Paketes
\hypersetup{%    
	pdftitle	= {\dctitle}, %
	pdfsubject	= {\dcsubject, \dcdate}, %
	pdfauthor	= {\dcauthornameriren \dcauthoremailriren \dcauthornameewie
	\dcauthoremailewie},
	%
	pdfkeywords	= {\dckeywords}, %
	pdfcreator	= {pdfTeX with Hyperref and Thumbpdf}, %
	pdfproducer	= {LaTeX, hyperref, thumbpdf}, %
	% weitere PDF-Einstellungen in hyperref.cfg
}
%%======================================================================
