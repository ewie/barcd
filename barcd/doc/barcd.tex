%======================================================================
%   Vorlage
%======================================================================
%   $Id$
%   Matthias Kupfer
%======================================================================
%   Documentclass
%======================================================================
\documentclass[
%   draft,          % Entwurfsmodus: Bilder als Rahmen,
                % Überlängen werden deutlich markiert
%   10pt,           % 
    11pt,           % KOMA default
%   12pt,           % 
    a4paper,        % DIN A4
    twoside,        % Zweiseitig
    german,         %
%----------------------------------------------------------------------
% Die folgenden Befehle stammen aus dem KOMA Paket
    headsepline,        % Linie unter der Kopfzeile 
%   headnosepline,      %
%   foodsepline,        % Linie über Fussnote   
    footnosepline=false,        %
    automark,       % Kolumnentitel lebendig
%   bigheadings,        % default: Überschriften gross setzen
    normalheadings,     % Überschriften normal setzen
%   smallheadings,      % Überschriften eher klein setzen
%   pointlessnumbers,   % Keinen Punkt hinter die letzte Zahl 
                % eines Kapitels (auch bei Anhang)
%   chapterprefix,      % Kapiteluberschriften "Kapitel"
%   appendixprefix,     % Anhang
%   openright,      % Kapitel auf der rechten (ungeraden) Seite 
                % anfangen lassen
    openany,        % 
%   cleardoublestandard,    %
    cleardoubleplain,   %
%   cleardoubleempty,   %
    abstracton,     %
    idxtotoc,       % Index soll im Inhaltsverzeichnis auftauchen
    liststotoc,     %
    bibtotoc,       %
%   parskip,        % parskip-, parskip*, parskip+
%   halfparskip,        % halfparskip-, halfparskip* und halfparskip+
%   DIVclassic,
    BCOR8mm,        %%?? Bindungskorrektur: 
                % BCOR<Breite des Bindungsverlustes> 
]{scrreprt} 

%======================================================================
%   Bearbeitung sowohl mit LaTeX als auch mit pdfLaTeX ermoeglichen
%======================================================================
%\newif\ifpdf 
    %\ifx\pdfoutput\undefined 
    %\pdffalse          % we are not running pdflatex 
%\else 
    %\pdfoutput=1           % we are running pdflatex 
    %\pdftrue 
%\fi


%======================================================================
%   Verwendete Pakete
%======================================================================

\usepackage[ngerman]{babel}     % Sprachen

%% Latex mit deutschen Umlauten:
%% http://www.cs.albany.edu/~herrmann/latex_umlaute/
\usepackage[T1]{fontenc}    % EC-Schriften verwenden (vs. DC) da 8-Bit
                % EC-Schriften als T1-kodierten CM-Schriften
                % European/Ext.-Computer-Modern-(EC)-Schriften
                % Umlaute, Anführungszeichen ...
                % => Umlauten koennen richtig getrennt werden
                % FAQ 5.3.2

%\usepackage{ucs}       % Eingabe von ü,ö,ä,ß mit UTF-8 erlaubt
\usepackage[utf8]{inputenc} % (auch in Include-Dateien)
                % ! Zeichensatz der Dateien als UTF-8 !
                                
\usepackage{ae,aecompl}     % virtuelle-CM-Fonts
                % da EC nicht als PostScript-(Type-1) verfuegbar
                % => keine echten Umlaute im PDF-Dokumen 
                %(Problem bei Suche)
                % By loading the ae package (\usepackage{ae}), 
                % you loose some characters as mentioned in 
                % README. 
                % The package aecompl by Denis Roegel restores
                % these characters which are taken from the ec 
                % fonts. If you use pdftex, you will get these 
                % characters as bitmaps, but this might be 
                % better than not having them at all.

%\usepackage{times, mathptm}    % TimesNewRoman Schrift (Acrobat Reader Fonts), 
                % dazu braucht man auch den entsprechenden 
                % Zeichensatz für den Math-Mode
%\usepackage{pslatex}       % ? mathematische Formeln mit Standard 
                % Postscript Fonts gesetzt
            % Paket     Roman         Serifenlos  Typewriter
%\usepackage{times} % -----------------------------------------------
            % times     Times         Helvetica   Courier
            % palatino  Palatino      Helvetica   Courier
            % newcent   NewCenturySch AvantGarde  Courier
            % bookman   Bookman       AvantGarde  Courier
            % Diese Schriften sind die Standard-PostScript-Schriften
            % und in jedem Drucker verfügbar

\usepackage{
%   german,         % Deutsche Trennungen (ALTE Rechtschreibung), 
                % Anführungsstriche und mehr 
%   ngerman,        % Deutsche Trennungen (NEUE Rechtschreibung), 
                % Anführungsstriche und mehr
                %
%   acronym,        % Verwaltung von Abkuerzungen
%   bibgerm,        % Deutsche Bibliographie 
    calc,           % Erweiterung der arithmetischen Funktionen in 
                % LaTeX
                % wird verwendet um Titelseite zu zentrieren
    color,          % im Laufenden Text einfach mit \color{Farbe) zwischen den 
                % Farben umschalten, wobei Farbe einfach 
                % durch z.B. red, blue, black etc. ersetzt wird
                % \textcolor{farbe){Text)
%   epigraph,       % Zitat am Kapitelanfang
%   fancyhdr,       % Kopf- und Fußzeilen von Dokumenten frei 
                % gestalten
    fancybox,       % shadowbox, doublebox, ovalbox, Ovalbox 
    fancyvrb,       % verbatim Erweiterung:
    float,          % Positionierung von Gleitobjekten genau an der Stelle, wo man
                % 'figure'- oder 'table'-Umgebung die 
                % Positionierung [H] gesetzt werden
%   glosstex,       % Glossar und Abkürzungsverzeichnis
    mdwlist,        % compact list: itemize* ..
    scrdate,        % \todaysname 
    scrtime,        % \thistime
    scrpage2,       % Kopf- und Fußzeilen flexibel gestalten
                %
%   moreverb,       % verbatim-ähnlich: boxedverbatim, listing
%   verbatim,       % Darstellung von "Text, wie er eingegeben wird"
                %
%   lscape,         % Erstellt eine um 90% gedrehte *neue* Seite
%   textcomp,       % Sonderzeichen
%   booktabs,       % Tabellenlinien
%   longtable,      % Tabellen > 1 Seite
%   supertabular,       % Tabellen > 1 Seite
    tabularx,       % Blocksatzspalten
%   ltxtable,       % tabularx + longtable
%   multicol,       % mehrspaltige Zeilen
%   varioref,       % einheitliche Verweise
%   endnotes,       % Fussnoten -> Endnoten
%   rotating,       % sidewaystable und sidewaysfigure
%   natbib,         % Bibliographie ohne Klammer etc.
%   marvosym,       % Euro etc.
}

%\usepackage{pstricks}
%\usepackage{listings}
%\usepackage{wasysym}

%\usepackage[german, first, bottomafter]{draftcopy}

%\usepackage{setspace}  % Durchschuß, Zeilenabstand
%\doublespace       % doppelzeilig oder
%\onehalfspacing    % anderthalbzeilig

% Für schöne Darstellung von Algorithmen
%\usepackage[german, algoruled, algochapter]{algorithm2e}
%\usepackage{algorithmic}
%\usepackage[chapter]{algorithm}
%\floatname{algorithm}{Algorithmus}

%======================================================================
%   Bilder, Links
%======================================================================

\usepackage{graphicx}
%\graphicspath{{../img/}}   % Angabe der Pfade, wo die Grafiken liegen; 
                % mehrere Pfade sind möglich

\usepackage{thumbpdf}
\usepackage{hyperref}


%======================================================================
%   user.sty
%======================================================================

\usepackage{user}   % Makros in user.sty 
            % \epsinc{bild.eps}{scale=1}{Bildunterschrift}
            % \missing{Beweis fehlt noch}
            % \comment{Ein Kommentar}

%======================================================================
%   Einstellungen
%======================================================================

%\typearea{11}      % Satzspiegel neu konstruieren (KOMA)
            %      10pt 11pt 12pt 
            % DIV : 8   10   12  

\pagestyle{scrheadings} % Standart  Kopf- und Fußzeile
\setkomafont{pagehead}{\small\scshape}

% ---------------------------------------------------------------------
%\setcounter{secnumdepth}{2}
%\setcounter{chapter}{-1}
\setcounter{tocdepth}{3}

% ---------------------------------------------------------------------
%\sloppy        % weniger Worttrennungen, größere Wortabstände
\fussy          % viele Worttrennungen, "schönere" Wortabstände

% ---------------------------------------------------------------------
%\flushbottom           % Ausrichtung der Seitenenden jeweils auf 
                % gleicher Höhe

% ---------------------------------------------------------------------
%\sloppypar     % Das hier relaxt die Einstellungen zum Wortabstand 
            % extrem. Damit ragen keine Worte über den rechten 
            % Zeilenabstand hinaus. Dafür muß stärker auf
            % Wortabstand geachtet werden, der kann dann ziemlich 
            % groß werden. Man erhält aber keine Meldung mehr 
            % über underfull boxes.

%Hiermit kann man das gleiche mit weniger Holzhammer erreichen:
%\setlength{\tolerance}{2000}           % Strafpunkt für Zeilenumbruch
%\setlength{\emergencystretch}{3pt}     % Soweit dürfen einzelne Worte mehr 
                    % auseinandergezogen werden
%\setlength{\hfuzz}{1pt}                % Macht den rechten Rand um bis zu 1pt 
                    % flatterig.

% ---------------------------------------------------------------------
%Vermeiden einzelner Zeilen am Ende einer Seite oder oben auf einer neuen Seite
%\clubpenalty10000
%\widowpenalty10000

% ----------------------------------------------------------------------
% neue Umgebungen für verwendete Sätze und Beispiele
\newtheorem{bsp}{Beispiel}[chapter]
\newtheorem{satz}{Satz}[chapter]

%======================================================================
%   includeonly
%======================================================================

\includeonly{       % Gibt an, welche Dateien der include-Befehl 
            % tatsächlich einfuegen darf.
  metadaten     % Variablen setzen
  ,titel        % Titelseite, Zusammenfassung und Inhaltsverzeichnis
% Alle per include einzulesenden Dateien müssen hier angegeben sein!
%  ,anleitung        % Anleitung zur Nutzung der Vorlage
}

% Aufteilung der Arbeit in folgende Bestandteile in angeg. Reihenfolge
%
% Titelseite
% Bibliographische Beschreibung (Rückseite der Titelseite)
% (Aufgabenstellung)
% Danksagung
% Abstract
% Inhaltsverzeichnis
% Tabellenverzeichnis
% Abbildungsverzeichnis
% Einleitung
% Inhalt
% Zusammenfassung/Ausblick
% (Verzeichnis verwendeter Terme)
% Glossar (Abkürzungsverzeichnis)
% (Index)
% Literaturverzeichnis
% (Thesen)
% (Selbstständigkeitserklärung)
% Anhänge


%======================================================================
% * *   T E X T    S T A R T S    H E R E   * * * * * * * * * * * * * *
%======================================================================

\begin{document}
%======================================================================
%	Metadaten
%======================================================================
%	$Id$
%	Matthias Kupfer
%======================================================================

\newcommand{\dcsubject}{Teamorientierte Projektarbeit}
% z.B. (Diplom/Studien/Haus)arbeit, Praktikumsbericht, Studie, Beleg, 
% (Pro/Haupt/Ober)seminar, Seminar usw.
\newcommand{\dctitle}{Entwurf und Implementation von Verfahren zum Auffinden und Lesen von Strichcodes in Videodaten}
\newcommand{\dcsubtitle}{~} % Untertitel, falls erforderlich

\newcommand{\dcauthornameewie}{Erik Wienhold}
\newcommand{\dcauthornameriren}{René Richter}
\newcommand{\dcauthoremailewie}{ewie@hrz.tu-chemnitz.de}
\newcommand{\dcauthoremailriren}{riren@hrz.tu-chemnitz.de}
\newcommand{\dcauthors}{%
    René Richter \normalsize\texttt{riren@hrz.tu-chemnitz.de} \\
    Erik Wienhold \normalsize\texttt{ewie@hrz.tu-chemnitz.de} \\
}
\newcommand{\dcdate}{\today} 

\newcommand{\dcplace}{Chemnitz} % Ort, kann an der TU meist so bleiben
\newcommand{\dcuni}{Technische Universität \dcplace}
\newcommand{\dcdepart}{Informatik} % Fakultätsangabe
\newcommand{\dcprof}{Medieninformatik} % Angabe der Professur

\newcommand{\dcpruefer}{}% Prüfer der Arbeit
\newcommand{\dcadvisor}{Robert Manthey}% Betreuer der Arbeit

\newcommand{\dckeywords}{Liste,von,Stichworten,die,als,Schlagworte,geeignet,
sind}

%%======================================================================
% Einstellungen des Hyperref-Paketes
\hypersetup{%    
	pdftitle	= {\dctitle}, %
	pdfsubject	= {\dcsubject, \dcdate}, %
	pdfauthor	= {\dcauthornameriren \dcauthoremailriren \dcauthornameewie
	\dcauthoremailewie},
	%
	pdfkeywords	= {\dckeywords}, %
	pdfcreator	= {pdfTeX with Hyperref and Thumbpdf}, %
	pdfproducer	= {LaTeX, hyperref, thumbpdf}, %
	% weitere PDF-Einstellungen in hyperref.cfg
}
%%======================================================================
 % Variablen, \hypersetup, etc.

%======================================================================
%   Titelseite (Zusammenfassung und Inhaltsverzeichnis)
%======================================================================

%======================================================================
%	Titelseite 
%======================================================================
%	$Date:$
%	$Revision:$
%	Matthias Kupfer
%======================================================================

%%======================================================================
%% Schmutztitel
%%======================================================================
%\extratitle{
%	\usekomafont{sectioning}\mdseries 
%	\begin{center}
%		\Huge \dcsubject\\[1.5ex]
%		\hrule
%		\vspace*{\fill}
%		\includegraphics{TUC_deutsch_einzeile_CMYK}
%	\end{center}
%}

%%======================================================================
%% Titelkopf
%%======================================================================
\titlehead{
	\vspace*{-1.5cm}
	% Schriftfamilie wie alle Überschriften, aber nicht fett
	\usekomafont{sectioning}\mdseries 
	\begin{center}
		\raisebox{-1ex}{\includegraphics[scale=1.4]{TUC_deutsch_einzeile_CMYK}}\\
		\hrulefill \\[1em]
		{\Large\dcdepart}\\[0.5em] 
		\dcprof
	\end{center}
	\vspace*{1.5cm}
}

%%======================================================================
%% Subjekt
%%======================================================================
\subject{\bf\Huge\dcsubject}


%%======================================================================
%% Titel
%%======================================================================
\title{\sf\Large
	\dctitle
	\\
	\dcsubtitle
}

%%======================================================================
%% Autor des Dokumentes
%%======================================================================
%\author{\dcauthorfirstname~\dcauthorlastname}
\author{%
    \dcauthornameriren \\
    \texttt\dcauthoremailriren
    \and
    \dcauthornameewie \\
    \texttt\dcauthoremailewie
}
	
%%======================================================================
%% Ort, Datum
%%======================================================================
\date{\dcplace, den \dcdate
}

%%======================================================================
%% Publishers
%%======================================================================
\publishers{
	{\parbox{\textwidth-8em}{
		\begin{tabbing}
			{\bf Betreuer:}\quad\=\kill
			{\bf Prüfer:}	\>\dcpruefer\\
			{\bf Betreuer:}	\>\dcadvisor
		\end{tabbing}	
	}}
}

%%======================================================================
%% bibliografische Angaben
%%======================================================================
\lowertitleback{
%\textbf{\dcauthorlastname, \dcauthorfirstname}\\
\textbf{\dcauthornameriren\quad\dcauthornameewie}\\
\dctitle\\
\dcsubject,~\dcdepart\\
\dcuni,~\ifcase\month\or
  Januar\or Februar\or März\or April\or Mai\or Juni\or
    Juli\or August\or September\or Oktober\or November\or Dezember\fi
    ~\number\year
}

%%======================================================================
%% maketitle
%%======================================================================

\maketitle

%%======================================================================
%% Danksagung
%%======================================================================
\thispagestyle{empty}
\null\vfil
\begin{center}
\usekomafont{sectioning}\textbf{Danksagung}
\vspace{-.5em}\vspace{\parsep}
%%
%% Hier steht der Text für die Danksagung
%%
\end{center}
\par\vfil\null
\cleardoubleemptypage
  
%%======================================================================
%%      Kurzfassung / Abstract
%%======================================================================
\def\abstractname{Abstract} 	% Wenn der Text "Zusammenfassung" erscheinen 
				% soll, dann muß dies auskommentiert werden
				
\begin{abstract}
%%
%% Inhalt der Arbeit
%%
\end{abstract}

%%======================================================================
%%      Inhaltsverzeichnis
%%======================================================================
\cleardoubleemptypage
\pagenumbering{roman}
%\pdfbookmark{Inhaltsverzeichnis}{Inhaltsverzeichnis}
\tableofcontents

%%======================================================================
%%      Abbildungsverzeichnis
%%======================================================================
\cleardoublepage
\markboth{Abbildungsverzeichnis}{Abbildungsverzeichnis}
\listoffigures

%%======================================================================
%%      Tabellenverzeichnis
%%======================================================================
\cleardoublepage
\markboth{Tabellenverzeichnis}{Tabellenverzeichnis}
\listoftables

%%======================================================================
%%      Algorithmenverzeichnis
%%======================================================================
%\renewcommand{\listalgorithmname}{Algorithmenverzeichnis}
%\cleardoublepage
%\addcontentsline{toc}{chapter}{Algorithmenverzeichnis}
%\listofalgorithms

%%======================================================================
%%      Abkuerzungsverzeichnis
%%======================================================================
%\cleardoublepage
%\addcontentsline{toc}{chapter}{Abkürzungsverzeichnis}
%\markboth{Abk"urzungsverzeichnis}{Abk"urzungsverzeichnis}
%\def\listacronymname{Abk"urzungsverzeichnis}
%\printglosstex(acr)

%%======================================================================
%%      Ende
%%======================================================================
\cleardoublepage
\pagenumbering{arabic}
\setcounter{page}{1}
 % Titelseite, Zusammenfassung 
        % Inhaltsverzeichnis
        % Abkuerzungsverzeichnis
        % -> danach Seitennummerierung bei 1

%======================================================================
%   Inhalt per include
%======================================================================

\include{anleitung} % Datei, welche die Anleitung zur Nutzung der 
            % Vorlage enthält


%======================================================================
%   Glossar
%======================================================================

%\manualmark
%\addcontentsline{toc}{chapter}{Glossar}
%\markboth{Glossar}{Glossar}
%\def\glossaryname{Glossar}
%\printglosstex(glo)
%\cleardoublepage

%======================================================================
%   Literaturverzeichnis
%======================================================================
\manualmark
\markboth{Literaturverzeichnis}{Literaturverzeichnis}
\bibliographystyle{user}
\bibliography{literatur}
\cleardoublepage

%======================================================================
%   Thesen
%======================================================================
%\chapter{Thesen}
%\include{thesen}

%======================================================================
%   Selbstständigkeitserklärung
%======================================================================
\chapter*{Selbstständigkeitserklärung}

Hiermit erkläre ich, daß ich die vorliegende Arbeit
selbstständig angefertigt, nicht anderweitig zu Prüfungszwecken vorgelegt und
keine anderen als die angegebenen Hilfsmittel verwendet habe. Sämtliche 
wissentlich verwendete Textausschnitte, Zitate oder Inhalte anderer Verfasser 
wurden ausdrücklich als solche gekennzeichnet.\\[2ex]
\dcplace, den \dcdate\\[6ex]
\flushleft
\begin{tabular}{p{5cm} p{5cm}}\hline
\footnotesize \dcauthornameriren &
\footnotesize \dcauthornameewie
\end{tabular}

%======================================================================
%   Anhang
%======================================================================


%\part*{Anhang}
\cleardoublepage
\begin{appendix}
    
%\include{anhanga}

\end{appendix}

\end{document}